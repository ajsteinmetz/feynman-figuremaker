% Compile this using LuaLaTeX, not pdfLaTeX!
\documentclass{article}
\usepackage{tikz-feynman}
\tikzfeynmanset{compat=1.1.0}
\begin{document}

\hoffset=-1in
\voffset=-1in
\setbox0\hbox{\begin{tabular}{cccc}

% Manual placement diagram example
\begin{tikzpicture}
	\begin{feynman}
		\vertex (a) {\(\mu^{-}\)};
		\vertex [right=of a] (b);
		\vertex [above right=of b] (f1) {\(\nu_{\mu}\)};
		\vertex [below right=of b] (c);
		\vertex [above right=of c] (f2) {\(\overline \nu_{e}\)};
		\vertex [below right=of c] (f3) {\(e^{-}\)};
	\diagram* {
		(a) -- [fermion] (b) -- [fermion] (f1),
		(b) -- [boson, edge label'=\(W^{-}\)] (c),
		(c) -- [anti fermion] (f2),
		(c) -- [fermion] (f3),
		};
	\end{feynman}
\end{tikzpicture}

% Automatic placement diagram example
%\feynmandiagram [horizontal=a to b] {
%	i1 [particle=\(e^{-}\)] -- [fermion] a -- [fermion] i2 [particle=\(e^{+}\)],
%	a -- [photon, edge label=\(\gamma\), momentum'=\(k\)] b,
%	f1 [particle=\(\mu^{+}\)] -- [fermion] b -- [fermion] f2 [particle=\(\mu^{-}\)],
%	};

\end{tabular}}
\pageheight=\dimexpr\ht0+\dp0\relax
\pagewidth=\wd0
\shipout\box0
\stop